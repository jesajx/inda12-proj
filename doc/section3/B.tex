\subsubsection*{Scenario 1}

Användaren är en datalog på kth som går första året.
Fysikkunskaperna är goda.
Användaren vill simulera ett solsystem med ca 3 planeter
i omloppsbana runt en stjärna.

Programmet startas med ett fönster.
Vänstra delen av skärmen består av en meny.
Resten av skärmen är i i stort helt svart, bortsett
från lite text.

I menyn finns parametrar som t.ex. massa och radie att ställa in.
Användaren skriver in stjärnans egenskaper och höger-klickar
sedan i mitten av skärmen där det då skapas en ifylld cirkel.

Användaren ställer in och placerar nästa planet.
När användaren klickar håller hen höger musknapp
och drar åt det hållet initial hastighet ska vara.
Medans användaren drar med musen så syns en pil från planeten till
pekaren med text brevid om hur hög hastigheten blir när användaren väl släpper.

En streckad ellips visar hur omloppsbanan kommer bli.
När användaren släpper så börjar den nyskapade planeten röra sig
i en ellips runt stjärnan.
Användaren repeterar mönstret för 2--3 planeter men märker denna gång
att när hen håller på att placera (musknapp nertryckt och vektorn
för hastighet synes) en ny planet så pausas simulationen för att
underlätta placeringen.
Alla planeter börjar inte gå i omloppsbana som planerat,
men med några försök lyckas användaren.

\subsubsection*{Scenario 2}

Användaren har viss erfarenhet av datorer (t.ex. spel) och
fysikkunskaperna är minimala.
Programmet startar likt beskrivet i \textit{Scenario 1}.
Även denna användare ska skapa ett solsystem med en stjärna
och några planeter i omloppsbana runt den.

Användaren testar att klicka ut lite planeter utan att ändra några parametrar.
De rör sig inte så användaren försöker klicka och dra i en av dem.
Ett snabbt vänsterklick på planeten markerar den och ändrar i menyn
till vänster.

Drag med vänster musknapp flytter planetens position.
En cirkel visar hur planeten hamnar efter flyttningen och en linje dras
under dragningen mellan planetens nuvarande position och musen.
Vidare visas den nya positionens koordinater bredvid cirkeln.
Efter en planet flyttats börjar den däremot inte röra sig.

Drag med höger musknapp ger däremot en hastigheten i motsvarande riktning.
Alla planeter pausas medans användaren drar.
Från den den högerklicks-dragna planeten visas en pil till muspekaren
och bredvid visas en siffra.
Ju längre pilen är, ju större blir siffran - alltså måste det vara hastigheten.
Användaren släpper höger musknappen och planeten får fart - ut från skärmen.
Användaren försöker få skärmen att följa efter och
upptäcker att det går att pause med space-knappen,
att zooma med mushjulet och att förflytta skärmen med piltangenterna.

Planeten ges nu på samma sätt lite fart tillbaka - lite lägre denna gång.
Användaren måste dra flera gånger för att få planeten att
åka åt rätt håll - högerklicks-drag lägger till hastighet,
snarare än ställer in en hastigheten.

Användaren provar nu att markera planeten och 
ändra dessa parameterar i menyn till vänster.
Ökad radie verkar inte öka gravitationkraften,
men massa verkar göra det.

Användaren ökar planetens massan rejält och alla andra planter börjar
röra sig snabbare och snabbare mot den.
Användaren märker att en linje ritas från den markerade planeten
i den riktning den förflyttar sig.

Vissa planeter missar den tunga planeten; de får större hastighet när
de kommer nära och slungas på så sätt in bakom planeten och iväg.
Andra planeter åker rakt in i planeten och med ett pling-ljud
studsar mot den och flyger iväg.

Användaren stannar den stora planeten genom att i menyn
sätta dess hastighet till noll.
Användaren testar därefter metodiskt att sätta ut planeter på olika avstånd
och hastigheter från den tunga planeten.
PLaneterna sätts ut med låga massor.

Användaren märker att linjen som ritas från markerade planeter i rörelse
ibland böjer sig - den visar alltså hur planeten kommer att röra
sig i framtiden.
Med hjälp av detta lyckas användaren efter några försök
få några planeter att åka i elliptiska banor runt den tunga planeten.


\subsubsection*{Scenario 3}

Scenariot börjar likt de två ovan.
Användaren börjar däremot genom att klicka på ''templates'' i vänstermenyn.
Hen får då upp ett fönster med en lista på olika förbyggda universum.
Höger halva av fönstret finns ett fält med en beskrivning av markerad
template.
Längst när på skärmen finns två knappar: ''cancel'' och ''load''.
Användaren väljer i listan passande system för uppgiften
(enkelt solsystem) och klickar på load.
På skärmen skapas ett fungerande solsystem.
Användaren zoomar ut med mushjulet för att får en bättre blick av systemet.

Användern provar ladda ett nytt solsystem, dubbelstjärnigt denna gång.
När användaren klickar på load tas föregående solsystem bort och
ersätts med det dubbelstjärniga.
