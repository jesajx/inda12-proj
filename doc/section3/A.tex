
\subsubsection{Scenario 1}

Användaren är en datalog på kth som går första året.
Fysikkunskaperna är goda.
Användaren vill simulera ett solsystem med 3 planeter
i omloppsbana runt en stjärna.

Programmet startas med ett fönster.
Vänstra delen av skärmen består av en meny.
Längst ner i mitten av skärmen är en tickande
klocka (en visare men inga siffror).
Resten av skärmen är svart.

I menyn finns parametrar som t.ex. massa och radie att ställa in.
Användaren skriver in stjärnans egenskaper och klickar
sedan i mitten av skärmen där det då skapas en ifylld cirkel.

Användaren ställer in nästa planet och håller inne mus-knappen där
planeten ska skapas och drar sedan pekaren åt det hållet initial
hastighet ska vara.
Medans användaren drar med musen så syns en pil från planeten till
pekaren med text brevid om hur hög hastigheten blir när användaren väl släpper.

En streckad ellips visar hur omloppsbanan kommer bli.
När användaren släpper så börjar den nyskapade planeten röra sig
i en ellips runt stjärnan.
Användaren repeterar mönstret för 2--3 planeter men märker denna gång
att när hen håller på att placera (musknapp nertryckt och vektorn
för hastighet synes) en ny planet så pausas simulationen för att
underlätta placeringen.

\subsubsection{Scenario 2}

Användaren har viss erfarenhet av datorer (bl.a. spel) och
fysikkunskaperna är minimala.
Programmet startar likt beskrivet i Scenario 1.
Även denna användare ska skapa ett solsystem med en stjärna
och några planter i omloppsbana runt den.

Användaren testar att klicka ut lite planeter utan att ändra några parametrar.
De rör sig inte så användaren försöker dra i en av dem.
Planeterna och klockan nere i mitten pausas samtidigt,
planeten markeras och
en hastighetpil visas från den till muspekaren.
Användaren släpper musknappen och planeten får fart - ut från skärmen.
Användaren försöker få skärmen att följa efter och
upptäcker att det går att pause med space-knappen och
att zooma med mushjulet.

Planten ges nu lite fart tillbaka - lite lägre denna gång.

Användaren provar nu att ändra parameterarna på planeten.
Ökad storlek verkar inte ge gravitation, men massa verkar göra det.

Användaren ökar massan rejält planeten och alla andra planter börjar
röra sig snabbare och snabbare mot den.

Vissa planeter missar den stora planeten; då får större hastighet när
de kommer nära och slungas på så sätt förbi.
Andra planeter åker rakt in i planeten och så att de smälter samman.
Användaren noterar att massan hos den stora (fortfarande markerade)
planeten ökar i menyn till vänster.

Användaren stannar den stora planeten genom att sätta dess hastighet till
noll i menyn till vänster och
börjar sätta ut planeter en sträcka ifrån den
- med olika hastigheter, åt olika håll och med låga massor.

Efter några försök har användaren lyckats få några planeter att åka i en
elliptisk bana runt den stor planeten.

% TODO ev. upptäcker templatesystemet.
