
Hur vi svarade på ovanstående frågor:
\begin{enumerate}
    \item Förflyttningen av planeter har delats upp i system och komponenter
        som beskrivet i section~\ref{sec:design}.

        \vspace{6pt}

        Gravitationssystemet använder ett s.k. Quad Tree för att
        spara positioner och massor.
        Varje nod i trädet innehåller referenser till fyra under-träd, 
        den totala massan av alla planeter i trädet och dess under-träd
        och det s.k. masscentrum av massorna och planterna.
        Varje planet jämförs sedan med trädet.
        Om en nods masscentrum är tillräckligt långt bort kan
        dess mass-total och -centrum
        användas för att beräkna noden som en planet, snarare än att
        dyka ner och jämföra med varje planet - vilket annars skulle göras
        om nodens masscentrum är nära med jämförande planeten.

        \vspace{6pt}

        Kollisionsystemet använder också ett Quad Tree.
        Skillnaden är att Gravitationsträdet spar planeter endast i
        de s.k. löven (längst ner i trädet), medans kollisionträdet
        spar planeter i alla noder.

        Trädet delar in simulatorvärlden i kvadranter.
        Varje under-träd till en nod motsvararer in kvadrant inom
        den kvadrat noden motsvarar.

        En s.k. hastighetscirkel skapas för varje planet - 
        den har samma position som planeten, med dess radie
        är planetes radie adderat med längden av planetens hastighetvektor.

        Dessa cirklar passas sedan in i trädets rutnät - så djupt ner i
        trädet som möjligt.

        Varje planet kan sedan jämföras med alla planter djupar ner i trädet.

    \item Kollisioner hanteras som elastiska stötar.
        Kollisionsystemet har snabbats upp med ett Quad Tree som beskriver
        ovan. Detta är en stor förbättrning från att alla planeter
        jämförs med alla ($O(n^2)$), men programmet blir ändå långsamt med
        många aktiva planeter.
        Andra sätt har prövats men har inte visat sig vara snabbare.
        Bland annat har trädet skrivits om så att planeter endast sparas
        i löven, likt gravitationsträdet gör. Detta gör
        dock att samma planet hamnar på flera platser i trädet, vilket
        i slutändan gör att fler jämförelser görs, t.ex. mellan samma
        planeter flera gånger.
    \item Parallellisering har endast används i framtidssystemet som körs i en egen tråd.
        % TODO fysik på egen tråd?
    \item Framtidssystemet har implementerats, men utan kollisioner,
        då dessa gjorde det för långsamt.
\end{enumerate}

Nya frågor med svar:

\begin{description}
    \item[fråga] svar % todo
\end{description}
