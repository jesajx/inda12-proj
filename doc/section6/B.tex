
Hur vi svarade på ovanstående frågor:
\begin{enumerate}
    \item Förflyttningen av planeter har delats upp i system och komponenter
        som beskrivet i 5 Programdesign.

        \vspace{6pt}

        Gravitationssystemet använder den s.k.
        Barnes-Hut-simulations-algoritmen
        \footnote{http://en.wikipedia.org/wiki/Barnes-Hut\_simulation}
        \footnote{http://arborjs.org/docs/barnes-hut}
        \footnote{http://www.cs.princeton.edu/courses/archive/fall03/cs126/assignments/barnes-hut.html}
        för att beräkna accelerationer.
        algoritmen bygger på ett s.k. Quad Tree, som är likt ett
        binärt sökträd, men med fyra undernoder.
        Varje nod motsvarar dessutom en kvadrat i spelvärlden och
        dess undernoder kvadranter av denna.
        Planeter spars endast i löven av trädet, och varje löv
        får endast innehålla en planet.
        Varje nod spar den totala massan av alla planeter den och
        alla dess undernoder håller och beräknar masscentrum.

        När trädet har byggts jämförs varje planet med trädet.
        Om en nods masscentrum är tillräckligt nära planeten
        dyker algoritmen ner i trädet, annars kan noden beräknas
        som en egen planet.

        Accelerationen från en planet till en annan beräknas med
        formeln $a = G*M/d^2$, där $a$ är accelerationen, $G$
        gravitationskonstanten, $M$ motstående planets massa (dvs.
        dit accelerationen är riktad) och $d$ är avståndet mellan
        planeterna.

        Hastigheten uppdateras från massan likt \verb#v += a*t#
        och positionen från hastigheten likt \verb#p += v*t#.
        Där \verb#t# är tid sedan förra uppdateringen.

        \vspace{12pt}

        Kollisionsystemet använder en s.k.
        \textit{Sweep and Prune}-algoritm (SAP)
        \footnote{http://jitter-physics.com/wordpress/?tag=sweep-and-prune}
        \footnote{http://www.codercorner.com/SAP.pdf}.


	
        Till varje planet skapas en \textit{hastighetscirkel} - 
        den har samma position som planeten, men dess radie
        är planetes radie adderat med längden av planetens hastighetvektor,
        likt: $c.r = p.r + p.v * t$.
        Cirkeln motsvarar alltså det område planeten kan påverka
        inom närmaste uppdatering.

        HastighetsCirkeln passas sedan in i en fyrkant - dvs.
        max och min värden beräknas i x- och y-led.
        Hastighetscirklarna sorteras sedan efter deras x-min.

        Det är sedan lätt att kontrollera vilka cirklar som överlappar
        i x-led.

        Om två cirklar överlappar i x-led, kontrolleras de i y-led.
        Om de även där överlappar kontrolleras om och hur
        planeterna faktiskt kolliderar - genom beräkna tiderna de kolliderar.

        Alla kollisioner som sker inom $t$ tid samlas och sorteras.

        Den tidigaste kollisionen och de som inträffar \textit{samtidigt}
        (enl. passande epsilon-skillnad) tas ut ur listan.
        Alla planeter uppdateras till tiden vid kollisionen likt
        \verb#p += v * T#.

        Kollisionerna hanteras och motsvarande planeters hastighetscirklar
        och dyl. uppdateras. 

        På detta sätt utförs alltså \verb#p += v*t# stegvis och
        kollisionsystemet ersätter positionsystemet.

    \item Kollisioner hanteras som elastiska stötar.
        Kollisionsystemet har snabbats upp med SAP som beskrivet ovan.
        Två kollisionshanteringsätt hade provats innan:
        alla planter jämförs med alla ($O(n^2)$),
        och en trädliknande struktur lik den i gravitationsystemet
        (troligtvis $\Omega(n\log n$).
    \item Parallellisering har endast används i framtidssystemet.
        Uppdatering av accelerationer, hastigheter och positioner
        körs i lagom takt genom att beräkna med tidsskillnader
        ($t$ som nämnt ovan).
        Om  motsvarande system skulle parallelliseras skulle dett troligtvis
        göras internt inom systemet (dvs. systemet skapar trådar
        och väntar på att dessa ska bli klara), vilket inte säkert
        hade gjort systemet snabbare eftersom dessa beror till stor del
        på sekventiella beräkningar.
        Parallelliseringen hade därför troligtvis bestått av få
        trådar som exekverar en stor del av systemet var, eller
        av många trådar som exekverar en mycket liten del av systemet.

        I det tidigare hade det troligtvis körts för få trådar för
        att det skulle gå snabbare och den senarer troligtvis för
        många trådar för att det skulle gå snabbare (och variabler
        hade behövt låsas för att användas av alla).

    \item Framtidssystemet har implementerats, men utan kollisioner
        då dessa gjorde det för långsamt.
\end{enumerate}

