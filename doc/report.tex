\documentclass[a4paper, 11pt]{article}

\usepackage[swedish]{babel}
\usepackage[T1]{fontenc}
\usepackage[utf8]{inputenc}

\usepackage{fancyhdr}
\pagestyle{fancy}
\fancyhead[L]{Johannes Olegård/Anton Olsson}
\fancyhead[R]{proj.plan}
\fancyhead[C]{inda12}
\fancyfoot[C]{\thepage}
\renewcommand{\headrulewidth}{0pt}
\renewcommand{\footrulewidth}{0pt}

\title{Planetsimulator}
\author{Johannes Olegård \and Anton Olsson}

% width
\marginparwidth=0pt
\oddsidemargin=0pt
\textwidth=450pt

%height
\topmargin=0pt
\headsep=10pt
\textheight=650pt

%foot
\setlength{\skip\footins}{1cm}

\begin{document}

\maketitle

\section{Programbeskrivning}
\label{sec:progb}

Programmet ska använda opengl för det grafiska med hjälp av bibloteket libgdx. http://libgdx.badlogicgames.com/features.html
Artemis entity system kommer användas för att lättare separera koduppgifter till mindre klasser där varje klass gör en(1) enskild sak. http://gamadu.com/artemis/index.html
Artemis använder sig av komponenter och system, där komponenter håller data och systemen bearbetar dom. Ett system använder endast några få komponenter och ignorerar alla andra objekt utan dom komponenterna.

Planeterna ska ha följande komponenter
\begin{enumerate}
    \item Position
    \item Hastighet
    \item Acceleration
    \item Massa
    \item Storlek
\end{enumerate}

Gravitations systemet behöver använda \textit{massa} och \textit{acceleration} för att räkna ut gravitations krafterna mellan planeter.
Accelerations systemet behöver \textit{acceleration} och \textit{hastighet} för att förändra hastigheten.
Hastighets systemet behöver \textit{position} och \textit{hastighet} för att uppdatera positionen.
Rit systemet behöver \textit{position} och \textit{storlek} för att rita planeterna på skärmen.

\vspace{12pt}
Planen lämnades in i tid, men programmet blev fungerande ganska snart,
men kodandet fortsatte in i v20.

Första användartestet gjordes onsdag v19.
Rapporten påbörjades först lördag v19.


Arbetsfördelning varierade med i sin helhet skrev Johannes större
delen av fysiksystemen och
Anton större delen av grafiken och programgrunden samt framtidssystemet.
Båda skrev dock lite på allt.

Vi skrev ungefär lika mycket på planen och rapporten.


\vspace{12pt}
Vid testerna märkte vi att det kanske inte var rimligt med de svårare uppgiterna.
P.g.a. tidsbrist hann vi bara med en person och då programmet
fortfarande inte var helt klart upptäckte vi en del buggar
(dvs. inte bara design-fel).



\section{Användarbeskrivning}
\label{sec:anvb}

Programmet ska använda opengl för det grafiska med hjälp av bibloteket libgdx. http://libgdx.badlogicgames.com/features.html
Artemis entity system kommer användas för att lättare separera koduppgifter till mindre klasser där varje klass gör en(1) enskild sak. http://gamadu.com/artemis/index.html
Artemis använder sig av komponenter och system, där komponenter håller data och systemen bearbetar dom. Ett system använder endast några få komponenter och ignorerar alla andra objekt utan dom komponenterna.

Planeterna ska ha följande komponenter
\begin{enumerate}
    \item Position
    \item Hastighet
    \item Acceleration
    \item Massa
    \item Storlek
\end{enumerate}

Gravitations systemet behöver använda \textit{massa} och \textit{acceleration} för att räkna ut gravitations krafterna mellan planeter.
Accelerations systemet behöver \textit{acceleration} och \textit{hastighet} för att förändra hastigheten.
Hastighets systemet behöver \textit{position} och \textit{hastighet} för att uppdatera positionen.
Rit systemet behöver \textit{position} och \textit{storlek} för att rita planeterna på skärmen.

\vspace{12pt}
Planen lämnades in i tid, men programmet blev fungerande ganska snart,
men kodandet fortsatte in i v20.

Första användartestet gjordes onsdag v19.
Rapporten påbörjades först lördag v19.


Arbetsfördelning varierade med i sin helhet skrev Johannes större
delen av fysiksystemen och
Anton större delen av grafiken och programgrunden samt framtidssystemet.
Båda skrev dock lite på allt.

Vi skrev ungefär lika mycket på planen och rapporten.


\vspace{12pt}
Vid testerna märkte vi att det kanske inte var rimligt med de svårare uppgiterna.
P.g.a. tidsbrist hann vi bara med en person och då programmet
fortfarande inte var helt klart upptäckte vi en del buggar
(dvs. inte bara design-fel).



\section{Användarscenarier}
\label{sec:anvsc}

Programmet ska använda opengl för det grafiska med hjälp av bibloteket libgdx. http://libgdx.badlogicgames.com/features.html
Artemis entity system kommer användas för att lättare separera koduppgifter till mindre klasser där varje klass gör en(1) enskild sak. http://gamadu.com/artemis/index.html
Artemis använder sig av komponenter och system, där komponenter håller data och systemen bearbetar dom. Ett system använder endast några få komponenter och ignorerar alla andra objekt utan dom komponenterna.

Planeterna ska ha följande komponenter
\begin{enumerate}
    \item Position
    \item Hastighet
    \item Acceleration
    \item Massa
    \item Storlek
\end{enumerate}

Gravitations systemet behöver använda \textit{massa} och \textit{acceleration} för att räkna ut gravitations krafterna mellan planeter.
Accelerations systemet behöver \textit{acceleration} och \textit{hastighet} för att förändra hastigheten.
Hastighets systemet behöver \textit{position} och \textit{hastighet} för att uppdatera positionen.
Rit systemet behöver \textit{position} och \textit{storlek} för att rita planeterna på skärmen.

\vspace{12pt}
Planen lämnades in i tid, men programmet blev fungerande ganska snart,
men kodandet fortsatte in i v20.

Första användartestet gjordes onsdag v19.
Rapporten påbörjades först lördag v19.


Arbetsfördelning varierade med i sin helhet skrev Johannes större
delen av fysiksystemen och
Anton större delen av grafiken och programgrunden samt framtidssystemet.
Båda skrev dock lite på allt.

Vi skrev ungefär lika mycket på planen och rapporten.


\vspace{12pt}
Vid testerna märkte vi att det kanske inte var rimligt med de svårare uppgiterna.
P.g.a. tidsbrist hann vi bara med en person och då programmet
fortfarande inte var helt klart upptäckte vi en del buggar
(dvs. inte bara design-fel).



\section{Testplan}
\label{sec:testplan}

Programmet ska använda opengl för det grafiska med hjälp av bibloteket libgdx. http://libgdx.badlogicgames.com/features.html
Artemis entity system kommer användas för att lättare separera koduppgifter till mindre klasser där varje klass gör en(1) enskild sak. http://gamadu.com/artemis/index.html
Artemis använder sig av komponenter och system, där komponenter håller data och systemen bearbetar dom. Ett system använder endast några få komponenter och ignorerar alla andra objekt utan dom komponenterna.

Planeterna ska ha följande komponenter
\begin{enumerate}
    \item Position
    \item Hastighet
    \item Acceleration
    \item Massa
    \item Storlek
\end{enumerate}

Gravitations systemet behöver använda \textit{massa} och \textit{acceleration} för att räkna ut gravitations krafterna mellan planeter.
Accelerations systemet behöver \textit{acceleration} och \textit{hastighet} för att förändra hastigheten.
Hastighets systemet behöver \textit{position} och \textit{hastighet} för att uppdatera positionen.
Rit systemet behöver \textit{position} och \textit{storlek} för att rita planeterna på skärmen.

\vspace{12pt}
Planen lämnades in i tid, men programmet blev fungerande ganska snart,
men kodandet fortsatte in i v20.

Första användartestet gjordes onsdag v19.
Rapporten påbörjades först lördag v19.


Arbetsfördelning varierade med i sin helhet skrev Johannes större
delen av fysiksystemen och
Anton större delen av grafiken och programgrunden samt framtidssystemet.
Båda skrev dock lite på allt.

Vi skrev ungefär lika mycket på planen och rapporten.


\vspace{12pt}
Vid testerna märkte vi att det kanske inte var rimligt med de svårare uppgiterna.
P.g.a. tidsbrist hann vi bara med en person och då programmet
fortfarande inte var helt klart upptäckte vi en del buggar
(dvs. inte bara design-fel).



\section{Programdesign}
\label{sec:design}
Programmet ska använda opengl för det grafiska med hjälp av bibloteket libgdx. http://libgdx.badlogicgames.com/features.html
Artemis entity system kommer användas för att lättare separera koduppgifter till mindre klasser där varje klass gör en(1) enskild sak. http://gamadu.com/artemis/index.html
Artemis använder sig av komponenter och system, där komponenter håller data och systemen bearbetar dom. Ett system använder endast några få komponenter och ignorerar alla andra objekt utan dom komponenterna.

Planeterna ska ha följande komponenter
\begin{enumerate}
    \item Position
    \item Hastighet
    \item Acceleration
    \item Massa
    \item Storlek
\end{enumerate}

Gravitations systemet behöver använda \textit{massa} och \textit{acceleration} för att räkna ut gravitations krafterna mellan planeter.
Accelerations systemet behöver \textit{acceleration} och \textit{hastighet} för att förändra hastigheten.
Hastighets systemet behöver \textit{position} och \textit{hastighet} för att uppdatera positionen.
Rit systemet behöver \textit{position} och \textit{storlek} för att rita planeterna på skärmen.

\vspace{12pt}
Planen lämnades in i tid, men programmet blev fungerande ganska snart,
men kodandet fortsatte in i v20.

Första användartestet gjordes onsdag v19.
Rapporten påbörjades först lördag v19.


Arbetsfördelning varierade med i sin helhet skrev Johannes större
delen av fysiksystemen och
Anton större delen av grafiken och programgrunden samt framtidssystemet.
Båda skrev dock lite på allt.

Vi skrev ungefär lika mycket på planen och rapporten.


\vspace{12pt}
Vid testerna märkte vi att det kanske inte var rimligt med de svårare uppgiterna.
P.g.a. tidsbrist hann vi bara med en person och då programmet
fortfarande inte var helt klart upptäckte vi en del buggar
(dvs. inte bara design-fel).



\section{Tekniska frågor}
\label{sec:qa}
Programmet ska använda opengl för det grafiska med hjälp av bibloteket libgdx. http://libgdx.badlogicgames.com/features.html
Artemis entity system kommer användas för att lättare separera koduppgifter till mindre klasser där varje klass gör en(1) enskild sak. http://gamadu.com/artemis/index.html
Artemis använder sig av komponenter och system, där komponenter håller data och systemen bearbetar dom. Ett system använder endast några få komponenter och ignorerar alla andra objekt utan dom komponenterna.

Planeterna ska ha följande komponenter
\begin{enumerate}
    \item Position
    \item Hastighet
    \item Acceleration
    \item Massa
    \item Storlek
\end{enumerate}

Gravitations systemet behöver använda \textit{massa} och \textit{acceleration} för att räkna ut gravitations krafterna mellan planeter.
Accelerations systemet behöver \textit{acceleration} och \textit{hastighet} för att förändra hastigheten.
Hastighets systemet behöver \textit{position} och \textit{hastighet} för att uppdatera positionen.
Rit systemet behöver \textit{position} och \textit{storlek} för att rita planeterna på skärmen.

\vspace{12pt}
Planen lämnades in i tid, men programmet blev fungerande ganska snart,
men kodandet fortsatte in i v20.

Första användartestet gjordes onsdag v19.
Rapporten påbörjades först lördag v19.


Arbetsfördelning varierade med i sin helhet skrev Johannes större
delen av fysiksystemen och
Anton större delen av grafiken och programgrunden samt framtidssystemet.
Båda skrev dock lite på allt.

Vi skrev ungefär lika mycket på planen och rapporten.


\vspace{12pt}
Vid testerna märkte vi att det kanske inte var rimligt med de svårare uppgiterna.
P.g.a. tidsbrist hann vi bara med en person och då programmet
fortfarande inte var helt klart upptäckte vi en del buggar
(dvs. inte bara design-fel).



\section{Arbetsplan}
\label{sec:arbplan}
Programmet ska använda opengl för det grafiska med hjälp av bibloteket libgdx. http://libgdx.badlogicgames.com/features.html
Artemis entity system kommer användas för att lättare separera koduppgifter till mindre klasser där varje klass gör en(1) enskild sak. http://gamadu.com/artemis/index.html
Artemis använder sig av komponenter och system, där komponenter håller data och systemen bearbetar dom. Ett system använder endast några få komponenter och ignorerar alla andra objekt utan dom komponenterna.

Planeterna ska ha följande komponenter
\begin{enumerate}
    \item Position
    \item Hastighet
    \item Acceleration
    \item Massa
    \item Storlek
\end{enumerate}

Gravitations systemet behöver använda \textit{massa} och \textit{acceleration} för att räkna ut gravitations krafterna mellan planeter.
Accelerations systemet behöver \textit{acceleration} och \textit{hastighet} för att förändra hastigheten.
Hastighets systemet behöver \textit{position} och \textit{hastighet} för att uppdatera positionen.
Rit systemet behöver \textit{position} och \textit{storlek} för att rita planeterna på skärmen.

\vspace{12pt}
Planen lämnades in i tid, men programmet blev fungerande ganska snart,
men kodandet fortsatte in i v20.

Första användartestet gjordes onsdag v19.
Rapporten påbörjades först lördag v19.


Arbetsfördelning varierade med i sin helhet skrev Johannes större
delen av fysiksystemen och
Anton större delen av grafiken och programgrunden samt framtidssystemet.
Båda skrev dock lite på allt.

Vi skrev ungefär lika mycket på planen och rapporten.


\vspace{12pt}
Vid testerna märkte vi att det kanske inte var rimligt med de svårare uppgiterna.
P.g.a. tidsbrist hann vi bara med en person och då programmet
fortfarande inte var helt klart upptäckte vi en del buggar
(dvs. inte bara design-fel).



\section{Sammanfattning}
% TODO sammanfattning

\section{Kontroller} % TODO bättre namn

Här följer en beskrivning av snabbknappar och liknande i simulatorn.

Vänster musknapp används för att markera planeter.
Om shift-knappen hålls inne kan flera planeter markeras.
Vänster-klickas någonstans på skärmen där ingen planet är
och shift-knappen inte är nedtryckt, avmarkeras alla planeter.

När en eller flera planeter är markerade visar fälten i vänster-menyn
planeternas parametrar.
När flera planeter är markerade visas medelvärden för position
och hastighet.

Höger musknapp används för att placera planter.
Om en eller flera planeter är markerade används dragning av höger musknapp
för ge planeterna hastighet i motsvarande riktning.

Om planeten är markerad gör ett klick på \verb#T#-knappen för att
skärmen ska följa efter planeten.

Piltangenterna samt knapparna \verb#W#, \verb#A#, \verb#S# och \verb#D#
används för att förflytta skärmen i simulatvärlden.
Mushjulet kan användas för att zooma ut och in.

\verb#N#-knappen markerar \textit{nästa} (från simulatorn sett)
planet och flyttar skrämen till den.

\end{document}
