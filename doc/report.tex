\documentclass[a4paper, 11pt]{article}

\usepackage[swedish]{babel}
\usepackage[T1]{fontenc}
\usepackage[utf8]{inputenc}
\usepackage{hyperref}

\usepackage{fancyhdr}
\pagestyle{fancy}
\fancyhead[L]{Johannes Olegård,Anton Olsson}
\fancyhead[R]{proj.plan}
\fancyhead[C]{inda12}
\fancyfoot[C]{\thepage}
\renewcommand{\headrulewidth}{0pt}
\renewcommand{\footrulewidth}{0pt}

\title{Planetsimulator}
\author{Johannes Olegård \and Anton Olsson}

% width
\marginparwidth=0pt
\oddsidemargin=0pt
\textwidth=450pt

%height
\topmargin=0pt
\headsep=10pt
\textheight=650pt

%foot
\setlength{\skip\footins}{1cm}

\begin{document}

\maketitle

\section{Programbeskrivning}
\label{sec:progb}

\subsection{A}

\subsection{Scenarie 1}

Användaren är en datalog på kth som går första året.
Fysikkunskaperna är goda.
Användaren vill simulera ett solsystem med 3 planeter
i omloppsbana runt en stjärna.

Programmet startas och ett fönster öppnas i mitten av skärmen
med tomt solsystem och en meny kant till vänster där det finns
parametrar som t.ex. massa och radie att ställa in.
Användaren ställer in stjärnans egenskaper och klickar
sedan i mitten av skärmen där det då skapas en ifylld cirkel.

Användaren ställer in nästa planet och håller inne mus knappen där
planeten ska skapas och drar sedan pekaren åt det hållet initial
hastighet ska vara.
Medans användaren drar med musen så syns en pil från planeten till
pekaren med text brevid om hur hög hastigheten blir när användaren väl släpper.

En streckad ellips visar hur omloppsbanan kommer bli.
När användaren släpper så börjar den nyskapade planeten röra sig
i en ellips runt stjärnan.
Användaren repeterar mönstret för 2--3 planeter men märker denna gång
att när hen håller på att placera (musknapp nertryckt och vektorn
för hastighet synes) en ny planet så pausas simulationen för att
underlätta placeringen.

\subsection{Scenarie 2}

Användaren är en idiot.
Fysikkunskaperna är dåliga. % TODO ev. inte är del av användargruppen?

Användaren klickar ut många planter, solar, etc.
Dessa börjar kollidera och flyga utanför skärmen %TODO
Användaren tröttnar snart. % TODO hur undviker vi detta?
% TODO explosioner, supernovor och svarta hål kanske?
% TODO verktyg för att skapa omloppsbanor?
% TODO template-solsystem?



\subsection{B}
Planen lämnades in i tid och programmet blev fungerande ganska snart,
men kodandet fortsatte in i v20.

Användartestet gjordes onsdag v19.
Rapporten påbörjades först lördag v19.


Arbetsfördelning varierade men i sin helhet skrev Johannes större
delen av fysiksystemen (förutom framtidssystemet) och
Anton större delen av grafiken, programgrunden och framtidssystemet.
Båda skrev dock lite på allt.

Vi skrev ungefär lika mycket på planen och rapporten.



\subsection{C}
Det var inga större problem med de tekniska frågorna.
Kollisionsystemet blev inte så snabbt som förhoppats,
men fungerar ändå för ett ganska stort antal planeter.




\section{Användarbeskrivning}
\label{sec:anvb}

\subsection{A}

\subsection{Scenarie 1}

Användaren är en datalog på kth som går första året.
Fysikkunskaperna är goda.
Användaren vill simulera ett solsystem med 3 planeter
i omloppsbana runt en stjärna.

Programmet startas och ett fönster öppnas i mitten av skärmen
med tomt solsystem och en meny kant till vänster där det finns
parametrar som t.ex. massa och radie att ställa in.
Användaren ställer in stjärnans egenskaper och klickar
sedan i mitten av skärmen där det då skapas en ifylld cirkel.

Användaren ställer in nästa planet och håller inne mus knappen där
planeten ska skapas och drar sedan pekaren åt det hållet initial
hastighet ska vara.
Medans användaren drar med musen så syns en pil från planeten till
pekaren med text brevid om hur hög hastigheten blir när användaren väl släpper.

En streckad ellips visar hur omloppsbanan kommer bli.
När användaren släpper så börjar den nyskapade planeten röra sig
i en ellips runt stjärnan.
Användaren repeterar mönstret för 2--3 planeter men märker denna gång
att när hen håller på att placera (musknapp nertryckt och vektorn
för hastighet synes) en ny planet så pausas simulationen för att
underlätta placeringen.

\subsection{Scenarie 2}

Användaren är en idiot.
Fysikkunskaperna är dåliga. % TODO ev. inte är del av användargruppen?

Användaren klickar ut många planter, solar, etc.
Dessa börjar kollidera och flyga utanför skärmen %TODO
Användaren tröttnar snart. % TODO hur undviker vi detta?
% TODO explosioner, supernovor och svarta hål kanske?
% TODO verktyg för att skapa omloppsbanor?
% TODO template-solsystem?



\subsection{B}
Planen lämnades in i tid och programmet blev fungerande ganska snart,
men kodandet fortsatte in i v20.

Användartestet gjordes onsdag v19.
Rapporten påbörjades först lördag v19.


Arbetsfördelning varierade men i sin helhet skrev Johannes större
delen av fysiksystemen (förutom framtidssystemet) och
Anton större delen av grafiken, programgrunden och framtidssystemet.
Båda skrev dock lite på allt.

Vi skrev ungefär lika mycket på planen och rapporten.



\subsection{C}
Det var inga större problem med de tekniska frågorna.
Kollisionsystemet blev inte så snabbt som förhoppats,
men fungerar ändå för ett ganska stort antal planeter.




\section{Användarscenarier}
\label{sec:anvsc}

\subsection{A}

\subsection{Scenarie 1}

Användaren är en datalog på kth som går första året.
Fysikkunskaperna är goda.
Användaren vill simulera ett solsystem med 3 planeter
i omloppsbana runt en stjärna.

Programmet startas och ett fönster öppnas i mitten av skärmen
med tomt solsystem och en meny kant till vänster där det finns
parametrar som t.ex. massa och radie att ställa in.
Användaren ställer in stjärnans egenskaper och klickar
sedan i mitten av skärmen där det då skapas en ifylld cirkel.

Användaren ställer in nästa planet och håller inne mus knappen där
planeten ska skapas och drar sedan pekaren åt det hållet initial
hastighet ska vara.
Medans användaren drar med musen så syns en pil från planeten till
pekaren med text brevid om hur hög hastigheten blir när användaren väl släpper.

En streckad ellips visar hur omloppsbanan kommer bli.
När användaren släpper så börjar den nyskapade planeten röra sig
i en ellips runt stjärnan.
Användaren repeterar mönstret för 2--3 planeter men märker denna gång
att när hen håller på att placera (musknapp nertryckt och vektorn
för hastighet synes) en ny planet så pausas simulationen för att
underlätta placeringen.

\subsection{Scenarie 2}

Användaren är en idiot.
Fysikkunskaperna är dåliga. % TODO ev. inte är del av användargruppen?

Användaren klickar ut många planter, solar, etc.
Dessa börjar kollidera och flyga utanför skärmen %TODO
Användaren tröttnar snart. % TODO hur undviker vi detta?
% TODO explosioner, supernovor och svarta hål kanske?
% TODO verktyg för att skapa omloppsbanor?
% TODO template-solsystem?



\subsection{B}
Planen lämnades in i tid och programmet blev fungerande ganska snart,
men kodandet fortsatte in i v20.

Användartestet gjordes onsdag v19.
Rapporten påbörjades först lördag v19.


Arbetsfördelning varierade men i sin helhet skrev Johannes större
delen av fysiksystemen (förutom framtidssystemet) och
Anton större delen av grafiken, programgrunden och framtidssystemet.
Båda skrev dock lite på allt.

Vi skrev ungefär lika mycket på planen och rapporten.



\subsection{C}
Det var inga större problem med de tekniska frågorna.
Kollisionsystemet blev inte så snabbt som förhoppats,
men fungerar ändå för ett ganska stort antal planeter.




\section{Testplan}
\label{sec:testplan}

\subsection{A}

\subsection{Scenarie 1}

Användaren är en datalog på kth som går första året.
Fysikkunskaperna är goda.
Användaren vill simulera ett solsystem med 3 planeter
i omloppsbana runt en stjärna.

Programmet startas och ett fönster öppnas i mitten av skärmen
med tomt solsystem och en meny kant till vänster där det finns
parametrar som t.ex. massa och radie att ställa in.
Användaren ställer in stjärnans egenskaper och klickar
sedan i mitten av skärmen där det då skapas en ifylld cirkel.

Användaren ställer in nästa planet och håller inne mus knappen där
planeten ska skapas och drar sedan pekaren åt det hållet initial
hastighet ska vara.
Medans användaren drar med musen så syns en pil från planeten till
pekaren med text brevid om hur hög hastigheten blir när användaren väl släpper.

En streckad ellips visar hur omloppsbanan kommer bli.
När användaren släpper så börjar den nyskapade planeten röra sig
i en ellips runt stjärnan.
Användaren repeterar mönstret för 2--3 planeter men märker denna gång
att när hen håller på att placera (musknapp nertryckt och vektorn
för hastighet synes) en ny planet så pausas simulationen för att
underlätta placeringen.

\subsection{Scenarie 2}

Användaren är en idiot.
Fysikkunskaperna är dåliga. % TODO ev. inte är del av användargruppen?

Användaren klickar ut många planter, solar, etc.
Dessa börjar kollidera och flyga utanför skärmen %TODO
Användaren tröttnar snart. % TODO hur undviker vi detta?
% TODO explosioner, supernovor och svarta hål kanske?
% TODO verktyg för att skapa omloppsbanor?
% TODO template-solsystem?



\subsection{B}
Planen lämnades in i tid och programmet blev fungerande ganska snart,
men kodandet fortsatte in i v20.

Användartestet gjordes onsdag v19.
Rapporten påbörjades först lördag v19.


Arbetsfördelning varierade men i sin helhet skrev Johannes större
delen av fysiksystemen (förutom framtidssystemet) och
Anton större delen av grafiken, programgrunden och framtidssystemet.
Båda skrev dock lite på allt.

Vi skrev ungefär lika mycket på planen och rapporten.



\subsection{C}
Det var inga större problem med de tekniska frågorna.
Kollisionsystemet blev inte så snabbt som förhoppats,
men fungerar ändå för ett ganska stort antal planeter.




\section{Programdesign}
\label{sec:design}

\subsection{A}

\subsection{Scenarie 1}

Användaren är en datalog på kth som går första året.
Fysikkunskaperna är goda.
Användaren vill simulera ett solsystem med 3 planeter
i omloppsbana runt en stjärna.

Programmet startas och ett fönster öppnas i mitten av skärmen
med tomt solsystem och en meny kant till vänster där det finns
parametrar som t.ex. massa och radie att ställa in.
Användaren ställer in stjärnans egenskaper och klickar
sedan i mitten av skärmen där det då skapas en ifylld cirkel.

Användaren ställer in nästa planet och håller inne mus knappen där
planeten ska skapas och drar sedan pekaren åt det hållet initial
hastighet ska vara.
Medans användaren drar med musen så syns en pil från planeten till
pekaren med text brevid om hur hög hastigheten blir när användaren väl släpper.

En streckad ellips visar hur omloppsbanan kommer bli.
När användaren släpper så börjar den nyskapade planeten röra sig
i en ellips runt stjärnan.
Användaren repeterar mönstret för 2--3 planeter men märker denna gång
att när hen håller på att placera (musknapp nertryckt och vektorn
för hastighet synes) en ny planet så pausas simulationen för att
underlätta placeringen.

\subsection{Scenarie 2}

Användaren är en idiot.
Fysikkunskaperna är dåliga. % TODO ev. inte är del av användargruppen?

Användaren klickar ut många planter, solar, etc.
Dessa börjar kollidera och flyga utanför skärmen %TODO
Användaren tröttnar snart. % TODO hur undviker vi detta?
% TODO explosioner, supernovor och svarta hål kanske?
% TODO verktyg för att skapa omloppsbanor?
% TODO template-solsystem?



\subsection{B}
Planen lämnades in i tid och programmet blev fungerande ganska snart,
men kodandet fortsatte in i v20.

Användartestet gjordes onsdag v19.
Rapporten påbörjades först lördag v19.


Arbetsfördelning varierade men i sin helhet skrev Johannes större
delen av fysiksystemen (förutom framtidssystemet) och
Anton större delen av grafiken, programgrunden och framtidssystemet.
Båda skrev dock lite på allt.

Vi skrev ungefär lika mycket på planen och rapporten.



\subsection{C}
Det var inga större problem med de tekniska frågorna.
Kollisionsystemet blev inte så snabbt som förhoppats,
men fungerar ändå för ett ganska stort antal planeter.




\section{Tekniska frågor}
\label{sec:qa}

\subsection{A}

\subsection{Scenarie 1}

Användaren är en datalog på kth som går första året.
Fysikkunskaperna är goda.
Användaren vill simulera ett solsystem med 3 planeter
i omloppsbana runt en stjärna.

Programmet startas och ett fönster öppnas i mitten av skärmen
med tomt solsystem och en meny kant till vänster där det finns
parametrar som t.ex. massa och radie att ställa in.
Användaren ställer in stjärnans egenskaper och klickar
sedan i mitten av skärmen där det då skapas en ifylld cirkel.

Användaren ställer in nästa planet och håller inne mus knappen där
planeten ska skapas och drar sedan pekaren åt det hållet initial
hastighet ska vara.
Medans användaren drar med musen så syns en pil från planeten till
pekaren med text brevid om hur hög hastigheten blir när användaren väl släpper.

En streckad ellips visar hur omloppsbanan kommer bli.
När användaren släpper så börjar den nyskapade planeten röra sig
i en ellips runt stjärnan.
Användaren repeterar mönstret för 2--3 planeter men märker denna gång
att när hen håller på att placera (musknapp nertryckt och vektorn
för hastighet synes) en ny planet så pausas simulationen för att
underlätta placeringen.

\subsection{Scenarie 2}

Användaren är en idiot.
Fysikkunskaperna är dåliga. % TODO ev. inte är del av användargruppen?

Användaren klickar ut många planter, solar, etc.
Dessa börjar kollidera och flyga utanför skärmen %TODO
Användaren tröttnar snart. % TODO hur undviker vi detta?
% TODO explosioner, supernovor och svarta hål kanske?
% TODO verktyg för att skapa omloppsbanor?
% TODO template-solsystem?



\subsection{B}
Planen lämnades in i tid och programmet blev fungerande ganska snart,
men kodandet fortsatte in i v20.

Användartestet gjordes onsdag v19.
Rapporten påbörjades först lördag v19.


Arbetsfördelning varierade men i sin helhet skrev Johannes större
delen av fysiksystemen (förutom framtidssystemet) och
Anton större delen av grafiken, programgrunden och framtidssystemet.
Båda skrev dock lite på allt.

Vi skrev ungefär lika mycket på planen och rapporten.



\subsection{C}
Det var inga större problem med de tekniska frågorna.
Kollisionsystemet blev inte så snabbt som förhoppats,
men fungerar ändå för ett ganska stort antal planeter.




\section{Arbetsplan}
\label{sec:arbplan}

\subsection{A}

\subsection{Scenarie 1}

Användaren är en datalog på kth som går första året.
Fysikkunskaperna är goda.
Användaren vill simulera ett solsystem med 3 planeter
i omloppsbana runt en stjärna.

Programmet startas och ett fönster öppnas i mitten av skärmen
med tomt solsystem och en meny kant till vänster där det finns
parametrar som t.ex. massa och radie att ställa in.
Användaren ställer in stjärnans egenskaper och klickar
sedan i mitten av skärmen där det då skapas en ifylld cirkel.

Användaren ställer in nästa planet och håller inne mus knappen där
planeten ska skapas och drar sedan pekaren åt det hållet initial
hastighet ska vara.
Medans användaren drar med musen så syns en pil från planeten till
pekaren med text brevid om hur hög hastigheten blir när användaren väl släpper.

En streckad ellips visar hur omloppsbanan kommer bli.
När användaren släpper så börjar den nyskapade planeten röra sig
i en ellips runt stjärnan.
Användaren repeterar mönstret för 2--3 planeter men märker denna gång
att när hen håller på att placera (musknapp nertryckt och vektorn
för hastighet synes) en ny planet så pausas simulationen för att
underlätta placeringen.

\subsection{Scenarie 2}

Användaren är en idiot.
Fysikkunskaperna är dåliga. % TODO ev. inte är del av användargruppen?

Användaren klickar ut många planter, solar, etc.
Dessa börjar kollidera och flyga utanför skärmen %TODO
Användaren tröttnar snart. % TODO hur undviker vi detta?
% TODO explosioner, supernovor och svarta hål kanske?
% TODO verktyg för att skapa omloppsbanor?
% TODO template-solsystem?



\subsection{B}
Planen lämnades in i tid och programmet blev fungerande ganska snart,
men kodandet fortsatte in i v20.

Användartestet gjordes onsdag v19.
Rapporten påbörjades först lördag v19.


Arbetsfördelning varierade men i sin helhet skrev Johannes större
delen av fysiksystemen (förutom framtidssystemet) och
Anton större delen av grafiken, programgrunden och framtidssystemet.
Båda skrev dock lite på allt.

Vi skrev ungefär lika mycket på planen och rapporten.



\subsection{C}
Det var inga större problem med de tekniska frågorna.
Kollisionsystemet blev inte så snabbt som förhoppats,
men fungerar ändå för ett ganska stort antal planeter.




\section{Sammanfattning}

Av projektet har vi, förutom att ha fått en inblick i just de
bibloteken vi använde, lärt oss om spel-fysik och -rendering
och hur lite om hur dessa kan optimeras.

Vi har också fått en känsla för hur användarinteraktion bör
vara - t.ex. vilka knappar och musrörelser som är intuitivt.

Programmet kan byggas ut nästan hur mycket som helst,
och vi skulle bl.a. gärna vilja lägga till lättare sätt
för att skapa omloppsbanor, dvs. att användaren säger att
en planet ska orbitera en annan och att start-parametrarna
automatiskt justeras.

Mycket i spelet skulle också kunna förbättras och debuggas.
Johannes vill t.ex. fortsätta optimera kollisions- och gravitationssystemen.

Tillsist har vi också lärt oss i hur mycket nytta man
kan ha av användartestning där vi upptäckte flera saker vi inte tänkt på.

\end{document}
