\documentclass[a4paper, 11pt]{article}

\usepackage[swedish]{babel}
\usepackage[T1]{fontenc}
\usepackage[utf8]{inputenc}
\usepackage{hyperref}

\usepackage{fancyhdr}
\pagestyle{fancy}
\fancyhead[L]{Johannes Olegård,Anton Olsson}
\fancyhead[R]{proj.plan}
\fancyhead[C]{inda12}
\fancyfoot[C]{\thepage}
\renewcommand{\headrulewidth}{0pt}
\renewcommand{\footrulewidth}{0pt}

\title{Planetsimulator}
\author{Johannes Olegård \and Anton Olsson}

% width
\marginparwidth=0pt
\oddsidemargin=0pt
\textwidth=450pt

%height
\topmargin=0pt
\headsep=10pt
\textheight=650pt

%foot
\setlength{\skip\footins}{1cm}

\begin{document}

\maketitle

\section{Programbeskrivning}
\label{sec:progb}

\subsection{A}
Programmet ska använda opengl för det grafiska med hjälp av bibloteket libgdx. http://libgdx.badlogicgames.com/features.html
Artemis entity system kommer användas för att lättare separera koduppgifter till mindre klasser där varje klass gör en(1) enskild sak. http://gamadu.com/artemis/index.html
Artemis använder sig av komponenter och system, där komponenter håller data och systemen bearbetar dom. Ett system använder endast några få komponenter och ignorerar alla andra objekt utan dom komponenterna.

Planeterna ska ha följande komponenter
\begin{enumerate}
    \item Position
    \item Hastighet
    \item Acceleration
    \item Massa
    \item Storlek
\end{enumerate}

Gravitations systemet behöver använda \textit{massa} och \textit{acceleration} för att räkna ut gravitations krafterna mellan planeter.
Accelerations systemet behöver \textit{acceleration} och \textit{hastighet} för att förändra hastigheten.
Hastighets systemet behöver \textit{position} och \textit{hastighet} för att uppdatera positionen.
Rit systemet behöver \textit{position} och \textit{storlek} för att rita planeterna på skärmen.


\subsection{B}
Planen lämnades in i tid, men programmet blev fungerande ganska snart,
men kodandet fortsatte in i v20.

Första användartestet gjordes onsdag v19.
Rapporten påbörjades först lördag v19.


Arbetsfördelning varierade med i sin helhet skrev Johannes större
delen av fysiksystemen och
Anton större delen av grafiken och programgrunden samt framtidssystemet.
Båda skrev dock lite på allt.

Vi skrev ungefär lika mycket på planen och rapporten.



\subsection{C}
Vid testerna märkte vi att det kanske inte var rimligt med de svårare uppgiterna.
P.g.a. tidsbrist hann vi bara med en person och då programmet
fortfarande inte var helt klart upptäckte vi en del buggar
(dvs. inte bara design-fel).



\section{Användarbeskrivning}
\label{sec:anvb}

\subsection{A}
Programmet ska använda opengl för det grafiska med hjälp av bibloteket libgdx. http://libgdx.badlogicgames.com/features.html
Artemis entity system kommer användas för att lättare separera koduppgifter till mindre klasser där varje klass gör en(1) enskild sak. http://gamadu.com/artemis/index.html
Artemis använder sig av komponenter och system, där komponenter håller data och systemen bearbetar dom. Ett system använder endast några få komponenter och ignorerar alla andra objekt utan dom komponenterna.

Planeterna ska ha följande komponenter
\begin{enumerate}
    \item Position
    \item Hastighet
    \item Acceleration
    \item Massa
    \item Storlek
\end{enumerate}

Gravitations systemet behöver använda \textit{massa} och \textit{acceleration} för att räkna ut gravitations krafterna mellan planeter.
Accelerations systemet behöver \textit{acceleration} och \textit{hastighet} för att förändra hastigheten.
Hastighets systemet behöver \textit{position} och \textit{hastighet} för att uppdatera positionen.
Rit systemet behöver \textit{position} och \textit{storlek} för att rita planeterna på skärmen.


\subsection{B}
Planen lämnades in i tid, men programmet blev fungerande ganska snart,
men kodandet fortsatte in i v20.

Första användartestet gjordes onsdag v19.
Rapporten påbörjades först lördag v19.


Arbetsfördelning varierade med i sin helhet skrev Johannes större
delen av fysiksystemen och
Anton större delen av grafiken och programgrunden samt framtidssystemet.
Båda skrev dock lite på allt.

Vi skrev ungefär lika mycket på planen och rapporten.



\subsection{C}
Vid testerna märkte vi att det kanske inte var rimligt med de svårare uppgiterna.
P.g.a. tidsbrist hann vi bara med en person och då programmet
fortfarande inte var helt klart upptäckte vi en del buggar
(dvs. inte bara design-fel).



\section{Användarscenarier}
\label{sec:anvsc}

\subsection{A}
Programmet ska använda opengl för det grafiska med hjälp av bibloteket libgdx. http://libgdx.badlogicgames.com/features.html
Artemis entity system kommer användas för att lättare separera koduppgifter till mindre klasser där varje klass gör en(1) enskild sak. http://gamadu.com/artemis/index.html
Artemis använder sig av komponenter och system, där komponenter håller data och systemen bearbetar dom. Ett system använder endast några få komponenter och ignorerar alla andra objekt utan dom komponenterna.

Planeterna ska ha följande komponenter
\begin{enumerate}
    \item Position
    \item Hastighet
    \item Acceleration
    \item Massa
    \item Storlek
\end{enumerate}

Gravitations systemet behöver använda \textit{massa} och \textit{acceleration} för att räkna ut gravitations krafterna mellan planeter.
Accelerations systemet behöver \textit{acceleration} och \textit{hastighet} för att förändra hastigheten.
Hastighets systemet behöver \textit{position} och \textit{hastighet} för att uppdatera positionen.
Rit systemet behöver \textit{position} och \textit{storlek} för att rita planeterna på skärmen.


\subsection{B}
Planen lämnades in i tid, men programmet blev fungerande ganska snart,
men kodandet fortsatte in i v20.

Första användartestet gjordes onsdag v19.
Rapporten påbörjades först lördag v19.


Arbetsfördelning varierade med i sin helhet skrev Johannes större
delen av fysiksystemen och
Anton större delen av grafiken och programgrunden samt framtidssystemet.
Båda skrev dock lite på allt.

Vi skrev ungefär lika mycket på planen och rapporten.



\subsection{C}
Vid testerna märkte vi att det kanske inte var rimligt med de svårare uppgiterna.
P.g.a. tidsbrist hann vi bara med en person och då programmet
fortfarande inte var helt klart upptäckte vi en del buggar
(dvs. inte bara design-fel).



\section{Testplan}
\label{sec:testplan}

\subsection{A}
Programmet ska använda opengl för det grafiska med hjälp av bibloteket libgdx. http://libgdx.badlogicgames.com/features.html
Artemis entity system kommer användas för att lättare separera koduppgifter till mindre klasser där varje klass gör en(1) enskild sak. http://gamadu.com/artemis/index.html
Artemis använder sig av komponenter och system, där komponenter håller data och systemen bearbetar dom. Ett system använder endast några få komponenter och ignorerar alla andra objekt utan dom komponenterna.

Planeterna ska ha följande komponenter
\begin{enumerate}
    \item Position
    \item Hastighet
    \item Acceleration
    \item Massa
    \item Storlek
\end{enumerate}

Gravitations systemet behöver använda \textit{massa} och \textit{acceleration} för att räkna ut gravitations krafterna mellan planeter.
Accelerations systemet behöver \textit{acceleration} och \textit{hastighet} för att förändra hastigheten.
Hastighets systemet behöver \textit{position} och \textit{hastighet} för att uppdatera positionen.
Rit systemet behöver \textit{position} och \textit{storlek} för att rita planeterna på skärmen.


\subsection{B}
Planen lämnades in i tid, men programmet blev fungerande ganska snart,
men kodandet fortsatte in i v20.

Första användartestet gjordes onsdag v19.
Rapporten påbörjades först lördag v19.


Arbetsfördelning varierade med i sin helhet skrev Johannes större
delen av fysiksystemen och
Anton större delen av grafiken och programgrunden samt framtidssystemet.
Båda skrev dock lite på allt.

Vi skrev ungefär lika mycket på planen och rapporten.



\subsection{C}
Vid testerna märkte vi att det kanske inte var rimligt med de svårare uppgiterna.
P.g.a. tidsbrist hann vi bara med en person och då programmet
fortfarande inte var helt klart upptäckte vi en del buggar
(dvs. inte bara design-fel).



\section{Programdesign}
\label{sec:design}

\subsection{A}
Programmet ska använda opengl för det grafiska med hjälp av bibloteket libgdx. http://libgdx.badlogicgames.com/features.html
Artemis entity system kommer användas för att lättare separera koduppgifter till mindre klasser där varje klass gör en(1) enskild sak. http://gamadu.com/artemis/index.html
Artemis använder sig av komponenter och system, där komponenter håller data och systemen bearbetar dom. Ett system använder endast några få komponenter och ignorerar alla andra objekt utan dom komponenterna.

Planeterna ska ha följande komponenter
\begin{enumerate}
    \item Position
    \item Hastighet
    \item Acceleration
    \item Massa
    \item Storlek
\end{enumerate}

Gravitations systemet behöver använda \textit{massa} och \textit{acceleration} för att räkna ut gravitations krafterna mellan planeter.
Accelerations systemet behöver \textit{acceleration} och \textit{hastighet} för att förändra hastigheten.
Hastighets systemet behöver \textit{position} och \textit{hastighet} för att uppdatera positionen.
Rit systemet behöver \textit{position} och \textit{storlek} för att rita planeterna på skärmen.


\subsection{B}
Planen lämnades in i tid, men programmet blev fungerande ganska snart,
men kodandet fortsatte in i v20.

Första användartestet gjordes onsdag v19.
Rapporten påbörjades först lördag v19.


Arbetsfördelning varierade med i sin helhet skrev Johannes större
delen av fysiksystemen och
Anton större delen av grafiken och programgrunden samt framtidssystemet.
Båda skrev dock lite på allt.

Vi skrev ungefär lika mycket på planen och rapporten.



\subsection{C}
Vid testerna märkte vi att det kanske inte var rimligt med de svårare uppgiterna.
P.g.a. tidsbrist hann vi bara med en person och då programmet
fortfarande inte var helt klart upptäckte vi en del buggar
(dvs. inte bara design-fel).



\section{Tekniska frågor}
\label{sec:qa}

\subsection{A}
Programmet ska använda opengl för det grafiska med hjälp av bibloteket libgdx. http://libgdx.badlogicgames.com/features.html
Artemis entity system kommer användas för att lättare separera koduppgifter till mindre klasser där varje klass gör en(1) enskild sak. http://gamadu.com/artemis/index.html
Artemis använder sig av komponenter och system, där komponenter håller data och systemen bearbetar dom. Ett system använder endast några få komponenter och ignorerar alla andra objekt utan dom komponenterna.

Planeterna ska ha följande komponenter
\begin{enumerate}
    \item Position
    \item Hastighet
    \item Acceleration
    \item Massa
    \item Storlek
\end{enumerate}

Gravitations systemet behöver använda \textit{massa} och \textit{acceleration} för att räkna ut gravitations krafterna mellan planeter.
Accelerations systemet behöver \textit{acceleration} och \textit{hastighet} för att förändra hastigheten.
Hastighets systemet behöver \textit{position} och \textit{hastighet} för att uppdatera positionen.
Rit systemet behöver \textit{position} och \textit{storlek} för att rita planeterna på skärmen.


\subsection{B}
Planen lämnades in i tid, men programmet blev fungerande ganska snart,
men kodandet fortsatte in i v20.

Första användartestet gjordes onsdag v19.
Rapporten påbörjades först lördag v19.


Arbetsfördelning varierade med i sin helhet skrev Johannes större
delen av fysiksystemen och
Anton större delen av grafiken och programgrunden samt framtidssystemet.
Båda skrev dock lite på allt.

Vi skrev ungefär lika mycket på planen och rapporten.



\subsection{C}
Vid testerna märkte vi att det kanske inte var rimligt med de svårare uppgiterna.
P.g.a. tidsbrist hann vi bara med en person och då programmet
fortfarande inte var helt klart upptäckte vi en del buggar
(dvs. inte bara design-fel).



\section{Arbetsplan}
\label{sec:arbplan}

\subsection{A}
Programmet ska använda opengl för det grafiska med hjälp av bibloteket libgdx. http://libgdx.badlogicgames.com/features.html
Artemis entity system kommer användas för att lättare separera koduppgifter till mindre klasser där varje klass gör en(1) enskild sak. http://gamadu.com/artemis/index.html
Artemis använder sig av komponenter och system, där komponenter håller data och systemen bearbetar dom. Ett system använder endast några få komponenter och ignorerar alla andra objekt utan dom komponenterna.

Planeterna ska ha följande komponenter
\begin{enumerate}
    \item Position
    \item Hastighet
    \item Acceleration
    \item Massa
    \item Storlek
\end{enumerate}

Gravitations systemet behöver använda \textit{massa} och \textit{acceleration} för att räkna ut gravitations krafterna mellan planeter.
Accelerations systemet behöver \textit{acceleration} och \textit{hastighet} för att förändra hastigheten.
Hastighets systemet behöver \textit{position} och \textit{hastighet} för att uppdatera positionen.
Rit systemet behöver \textit{position} och \textit{storlek} för att rita planeterna på skärmen.


\subsection{B}
Planen lämnades in i tid, men programmet blev fungerande ganska snart,
men kodandet fortsatte in i v20.

Första användartestet gjordes onsdag v19.
Rapporten påbörjades först lördag v19.


Arbetsfördelning varierade med i sin helhet skrev Johannes större
delen av fysiksystemen och
Anton större delen av grafiken och programgrunden samt framtidssystemet.
Båda skrev dock lite på allt.

Vi skrev ungefär lika mycket på planen och rapporten.



\subsection{C}
Vid testerna märkte vi att det kanske inte var rimligt med de svårare uppgiterna.
P.g.a. tidsbrist hann vi bara med en person och då programmet
fortfarande inte var helt klart upptäckte vi en del buggar
(dvs. inte bara design-fel).



\section{Sammanfattning}

Av projektet har vi, förutom att ha fått en inblick i just de
bibloteken vi använde, lärt oss om spel-fysik och -rendering
och hur lite om hur dessa kan optimeras.

Vi har också fått en känsla för hur användainteraktion bör
vara - t.ex. vilka knappar och musrörelser som är intuitivt.

Programmet kan byggas ut nästan hur mycket som helst,
och vi skulle bl.a. gärna vilja lägga till lättare sätt
för att skapa omloppsbanor, dvs. att användaren säger att
en planet ska orbitera en annan och att start-parametrarna
automatiskt justeras.

Vi skulle också vilja

Mycket i spelet skulle också kunna förbättras och debuggas.
Johannes vill t.ex. fortsätta optimera kollisions- och gravitationssystemen.

\end{document}
