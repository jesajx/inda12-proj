Programmet använder de utomstående bibloteken libgdx och artemis.

Planeterna representeras av entities med följande komponenter.
\begin{itemize}
    \item Position (vektor)
    \item Hastighet (vektor)
    \item Acceleration (vektor)
    \item Massa
    \item Storlek (radie)
    \item Färg (RGBA)
\end{itemize}

Fysiksystemen är:
\begin{description}
    \item[Gravitationssystemet] som beräknar gravitationsaccelerationen mellan planeter
        och summerar dessa för varje planet.
        Systemet uppdaterar alltså accelerationskomponenten för planterna.
    \item[Hastighetssystemet] uppdaterar hastigheteskomponenten hos
        planeter med accelerationskomponenten.
    \item[Kollisionsystemet] undersöker om och när planter kommer kollidera
        inom närmaste "tick" och hanterar dessa som elastiskta stötar.
        Systemet är ansvarigt för att uppdatera positionskomponenten
        hos planeter.
    \item[Framtidssystemet] beräknar hur en markerad planet
        kommer röra sig inom närmsta framtid och ritar detta på skärmen.
    \item[Positionssystemet] används av framtidssystemet för att uppdatera
        planeters positioner med deras hastigheter. Kollisionsystemet
        är nämligen för långsamt för att framtidssystemet ska
        vara användbart.
\end{description}

Användargränssnitt-systemen är.
\begin{description}
    \item[Inputsystemet] Hanterar markeringar, dragningar,
        zoomning och liknande av planeter.
        Systemet hanterar interaktion med simulator-världen.
    \item[Planet-ritar-systemet] Ritar planeter i simulatorvärlden.
    \item[Gränssnittssystemet] Hanterar vänstermenyn.
    \item[Template-gränsnittssystemet] Hanterar template-menyn.
    \item[Hjälp-gränsnittssystemet] Hanterar hjälp-menyn.
\end{description}

