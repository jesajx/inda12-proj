Vi testar programmet när det blivit stabilt och de flesta funktioner är fungerande genom användartestning.

Användartestningen skedde genom att ta tag i lämplig kth student efter en föreläsning och försöka få denne att utföra följande uppgifter:
\begin{enumerate}
    \item Skapa ett solsystem med en planet och en tillhörande måne.
    \item Skapa ett dubbelstjärnigt solsystem eventuellt med
        en tillhörande planet.
\end{enumerate}

Vi tog en människa från inda12 som har goda fysik kunskaper och passar in på användar scenario 1.
Innan användaren började förklarade vi kort med ord (utan att visa)
vad vi ville att hen skulle försöka utföra.
Vi förklarade att det grafiska gränssnittet ännu inte är användarvänligt,
 vilket är anledningen till att vi utför testet, och att eventuella misslyckanden inte beror på användaren.

Vi anmärkte tydligt att programmet är under utveckling och
att det därför är det fullt möjligt att buggar gör uppgifterna omöjliga.
Vi bad användaren att säga högt vad hen tänker så att vi kan förstå vad som
behöver ändras/förtydligas för en användarvänlig design.

Efter ca 1 minut så blev det uppenbart att kontrollerna inte var intuitiva nog då användaren inte förstod hur han skulle göra. Vi förklarade kort vilka mus- och tangent-knappar som gör vad och lät användaren fortsätta.

TODO
för stor massa och det blev otydlig infinite bug.
framtidsyn slutade fungera korrekt, visade endast nuvarande hastighet.

Vi upptäckte bl.a. att även fast användaren hade goda fysik kunskaper så var det svårt att uppskatta vilka storleksordningar på massor som är lagom för simulatorn. 
