
Vi testar programmet framförallt genom användartestning.

Användartestningen ska ske genom att ta tag i lämplig person
i godtycklig kth korridor och få denne att utföra en eller flera
av följande uppgifter:
\begin{enumerate}
    \item Skapa ett solsystem med en planet och en tillhörande måne.
    \item Justera omloppsbanan på en yttre planet i ett existerande
        solsystem med 2 orbitaler så att planeten hamnar innanför den andra.
    \item Skapa ett dubbelstjärnigt solsystem eventuellt med
        en tillhörande planet.
    \item Skapa en lite galax.
        Många stjärnor i mitten och fler hela solsystem som kretsar runt dessa.
\end{enumerate}
De två senare uppgifterna är lite svårare och vi räknare inte med
att användarna klarar dessa. Däremot vill vi gärna se hur de försöker använda
programmet för att lösa problemen.

Innan användaren börjar ska vi kort förklara med ord (utan att visa)
vad vi vill att hen ska försöka utföra.
Vi ska också tydligt förklara att eventuella misslyckanden
beror på brister i det grafiska gränssnittet och
inte användarens bristande intelligens.
Vi behöver tydligt anmärka att programmet är under utveckling och
att det därför är det fullt möjligt att buggar gör uppgifterna omöjlig.
Det bör finnas otydligheter i det grafiska gränsnittet och vi ber därför
användaren att säga högt vad hen tänker så att vi kan förstå vad som
behöver ändras/förtydligas för en användarvänlig design.

