Arbetet pågår under vecka 17 t.o.m. 20.
Under denna tid är det 3 övningstillfällen:
fre 26 april (v17),
fre 3 maj (v18) och
ons 15 maj (v20).

Projektplanen har deadline vid den första övningen.
Slutrapporten och programmet har deadline vid den sista övningen.
Vid varje övningstillfälle ges en muntlig lägesrapport.

Vecka 17 spenderas på att skapa ett git-repository med ett kodskelett och
att skriva projektplanen.
Om det finns tid över läggs den på att till viss del att börja
fylla kodskelettet.

Vecka 18 ska användas till att skriva så mycket av programmet som möjligt.
% TODO detaljer

Om det finns tid över börjar användartestningarna v18 och fullgörs annars
v19.

Vi räknar med ett fullt fungerande debuggad och användartestat
programmet i mitten av v19, 1 maj.
Slutrapporten ska skrivas så mycket som möjligt parallellt
med arbetet, men framförallt slutet av vecka 19 används till rapporten.
Om det finns tid över kan det användas till debugging i programmet.

Vi räknar med att slutrapporten är klar i slutet av vecka 19, fre 11 maj.

Vecka 20 (totalt 2 dagar innan inlämning) användes endast om något
moment tog längre tid än förväntat.
