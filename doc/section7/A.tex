Arbetet pågår under vecka 17 t.o.m. 20.
Under denna tid är det 3 övningstillfällen:
fre 26 april (v17),
fre 3 maj (v18) och
ons 15 maj (v20).

Projektplanen har deadline vid den första övningen.
Slutrapporten och programmetkoden har deadline vid den sista övningen.
Vid varje övningstillfälle ges en muntlig lägesrapport.

Vecka 17 spenderas på att skapa ett git-repository med ett kodskelett och
att skriva projektplanen.
Om det finns tid över läggs den på att till viss del att börja
fylla kodskelettet.

Vecka 18 ska användas till att skriva så mycket av programmet som möjligt.
Om det finns tid över börjar användartestningarna v18 och fullgörs annars
i början av v19.

Vi räknar med ett fullt fungerande, debuggat och användartestat
program i mitten av v19, ons 1 maj.
Slutrapporten ska skrivas så mycket som möjligt parallellt
med resten av arbetet,
men framförallt slutet av vecka 19 används till rapporten.
Om det finns tid över kan det användas till
debugging och finslipning av programmet.

Vi räknar med att slutrapporten är klar i slutet av vecka 19, fre 11 maj.

Vecka 20 (totalt 2 dagar innan inlämning) används endast om något
moment tog längre tid än förväntat.

Vi har ingen specifik arbetsfördelning utan skriver båda i koden och rapporten
- däremot kanske inte i samma filer samtidigt.
Vi håller kontakt via internet och kan snabbt komma överens om vad som
behöver göras för tillfället och vem som gör vad.

